% !TEX encoding = UTF-8 Unicode
% !TEX TS-program = lualatex
% !BIB TS-program = biber

%%%%%%% The above first line is used to auto-configure those LaTeX
%%%%%%% friendly editors, such as TeXShop, TeXworks, TeXstudio 
%%%%%%% so as to use the UTF-8 encoding for editing the source
%%%%%%% files and to save them.

%%%%%%% For the same editors the second line specifies which 
%%%%%%% typesetting engine is to be used when hitting the editor's
%%%%%%% button that launches the typesetting job.
%%%%%%% You can chose between pdflatex, lualatex or xelatex; in general
%%%%%%% it is better to avoid using xelatex and to prefer lualatex.

%%%%%%% If and only if you want to produce an archivable document
%%%%%%% according to the ISO regulation 19005:2005-1b add the
%%%%%%% following code before the statement \documentclass, and
%%%%%%% adapt its contents to your particular thesis.

\begin{filecontents*}{\jobname.xmpdata}
\Author{Mario Rossi}
\Title{Writing Your Ph.D. Thesis with LaTeX}
\Subject{Doctoral dissertations in the SCUDO doctoral school}
\Keywords{PDF\sep
          PDF/A\sep
          ISO 19005\sep
          LaTeX\sep
          PhD Thesis\sep
          Engineering\sep
          SCUDO}
\Publisher{Politecnico di Torino}
\end{filecontents*}


\documentclass[%
   12pt,
   twoside,
   tipotesi=scudo,
   numerazioneromana,
   ]{toptesi}
 
%%%% Use the following package if and only if you want to produce
%%%% an archivable document according to standard PDF/A-1b
%%%% No need to load package imakeidx, because it has already been
%%%% loaded by the specific module toptesi-scudo.
\usepackage[a-1b]{pdfx}
%%%%%% Read the English documentation of TOPtesi in order to check
%%%%%% the special attention needed to produce ISO compliant
%%%%%% archivable files

%%%%

\ifPDFTeX
    \usepackage[utf8]{inputenc}% 
    \usepackage[T1]{fontenc}
\fi

\errorcontextlines=9% more information on the console in case of errors

%%%% Specify fonts here; chose one among these fonts by leaving just
%%%% one line without initial comment character.
%%%% With LuaLaTeX or XeLaTeX don't change fonts
\ifPDFTeX % using pdflatex
    \usepackage{lmodern} % Default
    %\usepackage{newtxtext,newtxmath}% Times eXtended for text and math
    %\usepackage{fourier}% Utopia, Helvetica and "monospace = ?"

\else % using lualatex (or xelatex}
    \usepackage{fontspec}
    \defaultfontfeatures{Ligatures=TeX}
    \setmainfont{Libertinus Serif}
    \setsansfont{Libertinus Sans}
    \setmonofont{Libertinus Mono}[Scale=MatchLowercase]
    \usepackage{unicode-math}% add special math stile option here
                             % for example [style=ISO]
% chose one math font by commenting the line of the undesired font
    \setmathfont{XITS Math}
\fi

\makeindex[intoc]% collect material to index

\usepackage{kantlipsum} % to produce dummy text

%%%%%%%%%%%%%%%%%%%%%%%%%%%%%%%%%%%%%%%%%%%%%%%%%%%%%%%%%%%%%%
% This is to be loaded as the end of the preamble if one wants
% to use hyperlinks and/or urs to be typed within the thesis,
% except that after loading hyperref very few commands may be
% issued. One is the \includeonly command with its list of files;
% other packages may be loaded after hyperref, only if their
% documentation says so; some of these packages, but they are
% not the only ones, involve Right to left languages or other
% oriental languages.
%
% Distinguish the hyperref call from the hyperref setup so as
% to avoid option clashes with other packages that may invoke 
% hyperref with different options.

\unless\ifcsname ver@hyperref.sty\endcsname\usepackage{hyperref}\fi
\hypersetup{%
    pdfpagemode={UseOutlines},
    bookmarksopen,
    pdfstartview={FitH},
    colorlinks,
    linkcolor={blue},
    citecolor={blue},
    urlcolor={blue}
  }
%%%%%%%%%%%%%%%%%%%%%%%%%%%%%%%%%%%%%%%%%%%%%%%%%%%%%%%%%%%%%%

%%%% The \includeonly argument should preferably written with
%%%% one name per line, so that by commenting or uncommenting
%%%% some lines a selective compilation may be executed.
\includeonly{%
Chapter1/chapter1,%
Chapter2/chapter2,%
Chapter3/chapter3,%
Appendix1/appendix1,%
Appendix2/appendix2%
}


\ifPDFTeX \usepackage{indentfirst}\fi
\raggedbottom
\begin{document}


% The contents of this ThesisTitlePage environment may be written
% in a configuration file named exactly the same as the thesis main
% file, but with extension .cfg. If similar commands with different
% data are written within this environment, such data prevail on
% those read from the configuration file.
\begin{ThesisTitlePage}
% Use the optional command to set a different School logo
% Its is possible to used this command several times; each time
% a new different  Institution logo  may be added. In general
% just the ScuDo logo is sufficient; for dissertations made in
% cooperation with the University of Turin, its logo may be added
% with a second instance of the \PhGschoolLog statement. With
% dissertations supported by the INRiM, this institution logo may
% be added. Such logos (in PDF format) may be obtained from the
% ScuDo web server.
\PhDschoolLogo{Logo-ScuDo-blu} % just the ScuDo logo
%\PhDschoolLogo{Logo-ScuDo-blu,Logo-INRIM} %for dissertations made in cooperation with INRIM
%\PhDschoolLogo{Logo-ScuDo-blu,Logo-INFN} %for dissertations made in cooperation with INFN
%\PhDschoolLogo{Logo-ScuDo-blu,Logo-UniTo} %for dissertations made in cooperation with the University of Turin
% Doctorate course name; mandatory
\ProgramName{Energy Enginering}
% Cycle ordinal number; optional.
% You can write 29.th, or 29\ap{th}, or 29\textsuperscript{th}, or ...
\CycleNumber{29.th}
% PhD candidate name; mandatory
\author{Mario Rossi}
% Dissertation title; mandatory
\title{Writing your Doctoral~Thesis\\with \LaTeX}
% Dissertation subtitle: optional.
% It might be useful only if the actual full title is too long.
\subtitle{This document is an example of what you can do\\with~the~TOPtesi class}
% The supervisor(s) label; optional; default value "Supervisor:".
% You can change it to plural as in this example, where the colon has
% been eliminated.
%\NSupervisor{Supervisor}{Supervisors}
%
% The SupervisorNumber may contain a value such as 0, 1, or any
% number higher than 1. If 0 is specified, no label is typeset
% over the supervisor(s) list; if 1 is specified then the singular
% form is used: if any value higher than 1 is specified, the plural
% form is used.
\SupervisorNumber{2}
% List of supervisors with academic title, name(s), surname(s),
% and function; mandatory
\SupervisorList{%
    Prof.~A.B., Supervisor\\
    Prof.~C.D. Co-supervisor}
% Name of the examining committee: optional. 
% Default value "Doctoral Examination Committee"
%\Nexaminationcommittee{Doctoral examination committee}
% List of the  examining committee members: mandatory if the above label
% is not empty.
\ExaminerList{%
Prof.~A.B., Referee, University of \dots\\
Prof.~C.D., Referee, University of \dots\\
Prof.~E.F., University of \dots\\
Prof.~G.H., University of \dots\\
Prof.~I.J., University of \dots}
% Name of the institution where the examination takes place; optional.
% Default value: "Politecnico di Torino"
%\Nlocation{Politecnico di Torino}
% Examination date: mandatory, although the way to write it is free.
\ExaminationDate{February 29, 2123}
% Disclaimer with signature; optional. Default text as
% in the following lines. 
\Disclaimer{%
\noindent I hereby declare that, the contents and organisation of this dissertation constitute my own original work and does not compromise in any way the rights of third parties, including those relating to the security of personal data.	
}
%\Signature{%
%\begin{flushright}
%\parbox{0.5\textwidth}{\centering
%\dotfill\\
%Mario Rossi\\
%Turin, February 29, 2123
%}
%\end{flushright}}
\end{ThesisTitlePage}
%%%%%%%%%%%%%%%% Everything else necessary in the thesis title
%%%%%%%%%%%%%%%% page and in the copyright page is supplied by
%%%%%%%%%%%%%%%% the default values.
%%%%%%%%%%%%%%%% If you enter an explicit disclaimer, you can
%%%%%%%%%%%%%%%% typeset other material before the disclaimer
%%%%%%%%%%%%%%%% formula; use the necessary spacing on order
%%%%%%%%%%%%%%%% to separate the formula from other text. 
%%%%%%%%%%%%%%%% 
%%%%%%%%%%%%%%%% For example you might want to write the formula
%%%%%%%%%%%%%%%% for a particular licence, provided the licence 
%%%%%%%%%%%%%%%% allows Open access; not necessarily all uses
%%%%%%%%%%%%%%%% of the thesis should be allowed, but the minimum
%%%%%%%%%%%%%%%% is the reading access.

%%%%%%%%%%%%%%%% The next two lines are metadata for a normal PDF file.
%%%%%%%%%%%%%%%% For ISO archivable PDF/A-1b metadata, they must
%%%%%%%%%%%%%%%% be entered only in the form used in the file
%%%%%%%%%%%%%%%% named in the filecontents* environment as shown
%%%%%%%%%%%%%%%% at the beginning of this file.

%\subject{How to typeset a doctoral thesis suitable for defence in almost any country and university.}
%\keywords{{pdfLaTeX} {LuaLaTeX} {XeLaTeX} {PhD doctoral programs} {PhD dissertation} {Politecnico di Torino}} 

\summary%\sommario

This is where you write your abstract \dots\ (Maximum 4000 characters, i.e. maximum two pages in normal sized font, typeset with the thesis layout).

The abstract environment is also available, but  \texttt{\string\summary} is preferred because it generates an un-numbered chapter. The abstract environment is more suitable for articles and two column typesetting without a separate title page.

\emptypage %  it works even without the classica option
\acknowledgements% or \ringraziamenti

And I would like to acknowledge \dots

Acknowledgements are mandatory when people outside the academic institution supported the development of the research that was performed in order to reach the conclusion of the doctorate program.

\begin{dedication}

\textit{I would like to dedicate this thesis to my loving parents} 

{\normalsize 
The dedication very seldom is a proper thing to do; in some countries it is very common, while in other countries it is done for imitation of other people habits. 

The sentence used above clearly is an example of something very common, but it is  useless. Of course we all love our beloved parents, but it is not necessary to ``engrave it in stone''.
\par}
\end{dedication}
%\end{dedica}

%%%%%%%%% Unless you want these two lists, comment the following line
\tablespagetrue\figurespagetrue % 

%%%%%%%%% Table of contents and optionally the tables and figures lists
\allcontents

\mainmatter %all the above is front matter; here begins the main matter

\include{Chapter1/chapter1}
\include{Chapter2/chapter2}
% !TEX root = ../toptesi-scudo-example.tex
% !TEX encoding = UTF-8 Unicode
%*****************************************************************
%*********************** Third chapter ***************************
%*****************************************************************

\chapter{My third chapter}

% ********************** Define Graphics Path ********************
    \graphicspath{{Chapter3/}}

You should break your thesis up into nice, bite-sized sections and subsections. \LaTeX\ automatically builds a table of contents by looking  the \verb|\chapter{}|, \verb|\section{}|  and \verb|\subsection{}| commands you write in the source.

The Table of Contents should only list the sections to three (3) levels. A \verb|chapter{}| is level zero (0). A \verb|\section{}| is level one (1) and so a \verb|\subsection{}| is level two (2). In your thesis it is likely that you will even use a \verb|subsubsection{}|, which is level three (3). The depth to which the table of contents is formatted is set within the \textbf{TOPtesi.cls} class. If you need this changed, you can do it in the example source file \texttt{\jobname.tex} model. 

\section{First section of the third chapter}
And now I begin my third chapter here \dots

\kant[1-3]


\subsection{First subsection in the first section}
\dots and some more 

\kant[4]

\subsection{Second subsection in the first section}
\dots and some more \dots

\kant[5]

\subsubsection{First subsub section in the second subsection}
\dots and some more in the first sub-sub section otherwise it
all looks the same doesn't it? Well we can add some text to it \dots


Remember: each level may contain sublevels, but the latter must be al least two, otherwise subsectioning is useless
\kant[6]

\subsubsection{Second subsub section in the second section}
And this indeed is another subsection, so they are at least two.

\subsection{Third subsection in the first section}
\dots and some more \dots

\subsubsection{First subsub section in the third subsection}
\dots and some more in the first sub-sub section otherwise it all looks the same doesn't it? well we can add some text to it and some more \dots

\kant[7]

\subsubsection{Second subsub section in the third subsection}
\dots and some more in the second sub-sub section otherwise it all looks the same doesn't it? well we can add some text to it \dots

\kant[8]

\section{Second section of the third chapter}
and here I write more \dots

\kant[10-12]

\section{In Closing}

You have reached the end of this mini-guide. You can now rename or overwrite this PDF file and begin writing the rest of your thesis. The easy work of setting up the structure and framework has been taken care of for you. It's now your job to fill it out!

\vspace{\baselineskip}

\begin{center}\Large\bfseries
Good luck and have fun!
\end{center}

% Numbered appendices remain in the main matter...
\appendix
% !TEX root = ../toptesi-scudo-example.tex
% !TEX encoding = UTF-8 Unicode
% ******************************* Thesis Appendix A 

\chapter{How to install \LaTeX} 
\label{Appendix1}

These installing instructions are typical, but who prepared this file did not check their validity; the author of this example uses a Mac with OS~X; he can confirm that the instructions given below for this platform are correct, but he cannot honestly guarantee the same correctness for the other platforms.



\section*{Windows OS}
%+++++++++++++++++++++++++++++++++++++++++++++++++++++++++
\subsection*{Complete TeXLive \TeX~distribution}
\begin{enumerate}
\item	Download the TeXLive ISO from\\
\href{https://www.tug.org/texlive/}{https://www.tug.org/texlive/}\\
and open it by simply double clicking on its name in an Explorer  window.
\item If you don't have Win8 or higher do the following.
\begin{enumerate}
\item	Download WinCDEmu  from \\
\href{http://wincdemu.sysprogs.org/download/}
{http://wincdemu.sysprogs.org/download/}
\item	To install Windows CD Emulator follow the instructions at\\
\href{http://wincdemu.sysprogs.org/tutorials/install/}
{http://wincdemu.sysprogs.org/tutorials/install/}
\item	Right click the iso and mount it using the WinCDEmu as shown in \\
\href{http://wincdemu.sysprogs.org/tutorials/mount/}{
http://wincdemu.sysprogs.org/tutorials/mount/}
\end{enumerate}
\item If you have Win8 or higher open your the ISO image as if it were a real mounted disk, and run setup.pl.
\end{enumerate}

%++++++++++++++++++++++++++++++++++++++++++++++++++++++
\subsection*{Complete MikTeX - \TeX~distribution}
\begin{enumerate}
\item	Download Complete-MiK\TeX\ (32bit or 64bit) from\\
\href{http://miktex.org/download}{http://miktex.org/download}
\item	Run the installer 
\end{enumerate}

\subsection*{Textudio - \TeX~editor}
\begin{enumerate}
\item	Download TexStudio from\\
\href{http://texstudio.sourceforge.net/\#downloads}
{http://texstudio.sourceforge.net/\#downloads} 
\item	Run the installer
\end{enumerate}
%==========================================================
\section*{Mac OS X}
\subsection*{MacTeX - \TeX~distribution}
\begin{enumerate}
\item	Download the file from\\
\href{https://www.tug.org/mactex/}{https://www.tug.org/mactex/}
\item	Double click to run the installer and answer its questions. It does the entire configuration, sit back and relax.
\end{enumerate}

\subsection*{TeXShop or TexStudio - \TeX~editors}
\begin{enumerate}
\item Installing MacTeX gives you the opportunity to work with a Mac specific \TeX~editor, TeXShop; double click on its app in \texttt{\string~/Library/TeX/}; configure the launch bar to keep its icon to remain in the launch bar; set the TeXShop Preferences so as to have the UTF-8 encoding as the default one for editing and saving source files.
\item	If you are accustomed to different styled editors, download TexStudio from\\
\href{http://texstudio.sourceforge.net/\#downloads}
{http://texstudio.sourceforge.net/\#downloads} 
\item	Extract, Start, configure the launch bar so as to permanently keep its icon. 
\end{enumerate}

%===============================================
\section*{Unix/Linux}
\subsection*{Complete TeXLive - \TeX~ distribution}
\subsubsection*{Getting the distribution:}
TexLive can be downloaded from\\
  \href{http://www.tug.org/texlive/acquire-netinstall.html}
  {http://www.tug.org/texlive/acquire-netinstall.html}. 
Or a TexLive ISO file may be downloaded from the same location.
Follow the instructions given in the same Web site; Linux distributions are too different to give here a single  set of instructions valid for any incarnation of Linux. 

For Debian compliant Linux versions see the next section.

%\begin{description}
%  \item[Installation]~
%    \begin{enumerate}
%      \item[]	Mount the ISO file in the mnt directory
%        \begin{Verbatim}
%        mount -t iso9660 -o ro,loop,noauto /your/texlive####.iso /mnt
%        \end{Verbatim}
%    \end{enumerate}
%  \item[Install form ISO]~\\
%    Open the ISO file by double clicking on its icon; select 
%    \texttt{install-tl} and execute it.
%  \item[Installing]~
%    \begin{enumerate}
%      \item	Install wget on your OS (use rpm, apt-get or yum install)
%      \item	Run the installer script install-tl.
%      \begin{verbatim}
%      	cd /your/download/directory
%      	./install-tl
%      \end{verbatim}
%      \item	Enter command `i' for installation
%      
%      \item	Post-Installation configuration:\\
%      \href{http://www.tug.org/texlive/doc/texlive-en/texlive-en.html\#x1-320003.4.1}
%      {http://www.tug.org/texlive/doc/texlive-en/texlive-en.html\#x1-320003.4.1} 
%      \item	Set the path for the directory of TexLive binaries in your 
%      \texttt{.bashrc} file.
%    \end{enumerate}
%\end{description}
%%
%For Bourne-compatible shells (such as the bash shell) and using an Intel x86 GNU\discretionary{/}{}{/}Linux\footnote{The primitive command \texttt{\string\discretionary} inserts a discretionary line break; in this case it was inserted to allow a break after the slash sign. It is very unlikely that the final user of this template needs to use such tricks. In any case the user may read some advanced \TeX\ system guide, or the ultimate \TeX\ reference, the {\TeX}\-book itself, in oder to get the complete documentations about such native commands.} with a default directory setup as an example, the file to edit might be one of the following.
%\begin{description}
%  \item[For 32bit OS]~\\
%    \begin{Verbatim}
%    edit $~/.bashrc file and add following lines
%    PATH=/usr/local/texlive/2011/bin/i386-linux:$PATH; 
%    export PATH 
%    MANPATH=/usr/local/texlive/2011/texmf/doc/man:$MANPATH;
%    export MANPATH 
%    INFOPATH=/usr/local/texlive/2011/texmf/doc/info:$INFOPATH;
%    export INFOPATH
%    \end{Verbatim}
%  \item[For 64bit OS]~\\
%    \begin{Verbatim}
%    edit $~/.bashrc file and add following lines
%    PATH=/usr/local/texlive/2011/bin/x86_64-linux:$PATH;
%    export PATH 
%    MANPATH=/usr/local/texlive/2011/texmf/doc/man:$MANPATH;
%    export MANPATH 
%    INFOPATH=/usr/local/texlive/2011/texmf/doc/info:$INFOPATH;
%    export INFOPATH
%    \end{Verbatim}
%\end{description}
%%\end{enumerate}

\subsubsection*{Debian}
A Debian compliant TexLive is provided by most Linux operating systems; you can use (rpm, apt-get, yum, dots) to get TexLive packages; pay attention to download the complete set of different packages into which the Debian compliant Linux TexLive  distribution is subdivided. 
%
\begin{description}
  \item[Fedora/RedHat/CentOS:]~\\[-\baselineskip]
    \begin{verbatim} 
    sudo yum install texlive 
    sudo yum install psutils 
    \end{verbatim}
  \item[SUSE:]~\\[-1.5\baselineskip]
    \begin{verbatim}
    sudo zypper install texlive
    \end{verbatim}
  \item[Debian/Ubuntu:]~\\[-\baselineskip]
    \begin{verbatim} 
    sudo apt-get install texlive texlive-latex-extra 
    sudo apt-get install psutils
    \end{verbatim}
\end{description}

Pay attention to this substantial difference; TeXLive is a very lively maintained distribution; there are daily upgrades of existing packages and some new packages every week. The TeXLive distribution installed from a CTAN archive or mirror is updated almost every day; nobody needs to upgrade his/her installation everyday, but it is a good policy to do this simple operation (by means of the installed program \texttt{tlmgr} GUI) every week or so; twice a month is the suggested upgrading frequency.

The Debian compliant installation gets upgraded by the Debian consortium before being released to the users; this takes place every few months, in any case at least once a year. The Debian installation therefore lacks the \texttt{tlmgr} GUI and the user can only explore the Debian repositories to find out if there exists an updated TeXLive version.

In any case there is an article on TUGboat (the official magazine of the international \TeX\ Users Group) that explains how to install the CTAN TeXLive version (\cite{art:GregorioTUB-ubuntu}\footnote{In spite of being published in 2011, the article is till valid, even if sometimes a few details on Linux platforms have changed. You can download this article from this link: \url{https://www.tug.org/TUGboat/tb32-1/tb100gregorio.pdf}. An Italian version of this article can be downloaded from: \url{http://profs.sci.univr.it/~gregorio/texlive-ubuntu.pdf}}) on Linux platforms, with particular attention to the Debian compliant operating systems. This CTAN installation can live side by side with the Debian one; the former for actual typesetting, the latter for satisfying certain Debian dependencies. It is not mandatory to use the CTAN installation on Debian platforms, but it is strongly suggested in accordance with the different updating\slash upgrading policies followed by the CTAN maintainers compared to the Debian ones.





% !TEX root = ../toptesi-scudo-example.tex
% !TEX encoding = UTF-8 Unicode
% ******************************* Thesis Appendix B 

\chapter{Title of the second appendix}

\kant[1-4]% filler dummy text



\backmatter% here begins the back matter
% ... otherwise a single appendix may stay here
% \include{Appendix0/appendix0}

% Do not use this command if you did not prepare a nomenclature
% database by means od the suitable \nomenclature command and its
% arguments, as we did in chapter 2 of this example thesis.
\printnomencl

% In this example we use \nocite{*} in order to typeset the whole
% contents of the bibliographic database. Normally this is not
% necessary and it's better to let biber extract from the database
% only the cited works
\nocite{*}

\printbibliography[heading=bibintoc]

% Do not use this command if you did not set the \makeindex switch
% in the preamble.

\printindex

\end{document}

